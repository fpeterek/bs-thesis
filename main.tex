% Nejprve uvedeme tridu dokumentu s volbami
\documentclass[czech,bachelor,dept460,male,cpp,cpdeclaration]{diploma}
% Dalsi doplnujici baliky maker
\usepackage[autostyle=true,czech=quotes]{csquotes} % korektni sazba uvozovek, podpora pro balik biblatex
\usepackage[backend=biber, style=iso-numeric, alldates=iso]{biblatex} % bibliografie
\usepackage{dcolumn} % sloupce tabulky s ciselnymi hodnotami
\usepackage{subfig} % makra pro "podobrazky" a "podtabulky"

% Zadame pozadovane vstupy pro generovani titulnich stran.
\ThesisAuthor{Filip Peterek}

\CzechThesisTitle{Vývoj samoříditelné platformy}

\EnglishThesisTitle{Self-driving Platform Development}

\SubmissionDate{30. dubna 2021}

% Pokud nechceme nikomu dekovat makro zapoznamkujeme.
\Thanks{
	Chtěl bych poděkovat vedoucímu mé práce, Ing. Janu Gaurovi, nejen za pomoc s tvorbou práce a návrhem softwaru,
	ale především za možnost se projevit a pracovat na skutečném a zajímavém projektu. Dále chci poděkovat svým
	kolegům z práce, kteří mě naučili spoustu věcí, na které v akademickém prostředí není čas nebo příležitost.
}

% Zadame cestu a jmeno souboru ci nekolika souboru s digitalizovanou podobou zadani prace.
% Pokud toto makro zapoznamkujeme sazi se stranka s upozornenim.
\ThesisAssignmentImagePath{Figures/Assignment}

% Zadame soubor s digitalizovanou podobou prohlaseni autora zaverecne prace.
% Pokud toto makro zapoznamkujeme sazi se cisty text prohlaseni.
\AuthorDeclarationImageFile{Figures/AuthorDeclaration.jpg}


% Zadame soubor s digitalizovanou podobou souhlasu spolupracujici prav. nebo fyz. osoby.
% Pokud toto makro zapoznamkujeme sazi se cisty text souhlasu.
\CooperatingPersonsDeclarationImageFile{Figures/CoopPersonDeclaration.jpg}

\CzechAbstract{
	Tato práce se zabývá vývojem software pro samořiditelnou platformu vznikající na VŠB-TUO. Po dokončení by software
	měl být schopen na základě vstupů ze softwaru pro analýzu obrazu, GPS, Lidaru a řídící jednotky vozidla bezpečně
	navigovat vozidlo univerzitním kampusem.

	Řídící software je rozdělen do dvou komponent. První komponenta zajišťuje komunikaci s řídící jednotkou vozidla
	pomocí sběrnice CAN, na starosti má ovládání rychlosti vozidla a natáčení kol. Druhá komponenta má na starosti
	plánování cesty na základě výše vyjmenovaných vstupů. První komponentě poté zadá pouze požadovanou rychlost vozidla
	a natáčení kol, samotné posílání CAN zpráv však nemusí řešit. Komponenty mezi sebou komunikují pomocí protokolu TCP.
	K testování ovládacího software lze využít simulátor, který dokáže nasimulovat chování první komponenty. Pro řízení
	vozidla jsou v hojné míře využívány PID regulátory.

	V době odevzdání této práce ještě není software plně dokončený, jelikož se jedná o výzkumný projekt, který svým
	rozsahem přesahuje bakalářskou práci, a vývoj platformy bude dále pokračovat.
}

\CzechKeywords{
	plánování cesty, client-server komunikace, sběrnice CAN, PID regulátor
}

\EnglishAbstract{
	This thesis is focused on the development of software for a self-driving platform, which is being developed by
	the Technical University of Ostrava. Upon completion, the software should be capable of safely navigating an
	autonomous vehicle through the university campus, taking inputs from image analysis software, GPS, Lidar, and the
	vehicle's own controller unit.

	The software is split into two components. The first component handles communication with the vehicle's controller
	unit via a CAN bus. The second component then does all the path planning, taking into account the aforementioned
	inputs. The second component then asks the first component to maintain a certain speed or steer the vehicles,
	however, it doesn't have to handle CAN messages, including driving inputs or control checksums, by itself. The TCP
	protocol is used for communication between the components. The path planning software can be tested using
	a simulator, capable of mocking the CAN handling software, along with the entire vehicle. The simulator honors
	the TCP API, however, the vehicle physics are simplified. PID controllers are used in multiple parts of the
	software.

	At the time of publishing this paper, the software is still unfinished, as the self-driving platform is an academic
	research project, which outspans this bachelor's thesis in scope, and development of the platform will continue.
}

\EnglishKeywords{
	path planning, client-server communication, CAN bus, PID controller
}

\AddAcronym{CAN}{Controller Area Network}
\AddAcronym{PID}{Proporcionální-integrační-derivační}
\AddAcronym{TCP}{Transmission Control Protocol}
\AddAcronym{GPS}{Global Positioning System}
\AddAcronym{API}{Application Programming Interface}
\AddAcronym{XML}{Extensible Markup Language}
\AddAcronym{JSON}{JavaScript Object Notation}
\AddAcronym{TX}{Transmitter}
\AddAcronym{RX}{Receiver}
\AddAcronym{RPC}{Remote Procedure Call}

\addbibresource{biblatex-examples.bib}
\addbibresource{coffee.bib}

% Novy druh tabulkoveho sloupce, ve kterem jsou cisla zarovnana podle desetinne carky
\newcolumntype{d}[1]{D{,}{,}{#1}}


% Zacatek dokumentu
\begin{document}

% Nechame vysazet titulni strany.
\MakeTitlePages

% A nasleduje text zaverecne prace.
\section{Úvod}
\label{sec:Introduction}
Tato práce je součást většího výzkumného projektu. Cílem projektu je vývoj autonomního eletrifikovaného vozidla, 
od fyzického návrhu a zkonstruování vozidla, až po vývoj softwaru. Hotové vozidlo má být schopno plně autonomního pohybu
po univerzitním kampusu, kde má sloužit k rozvozu materiálu napříč budovami. Při tvorbě samořiditelné platformy je nutné
klást vysoký důraz na bezpečnost provozu, výsledek projektu při reálném využití nesmí ohrozit zdraví ani majetek 
univerzity nebo jejích návštěvníků. 

Samotná práce se zabývá vývojem řídícího softwaru vozidla. Software bude sloužit k navigaci vozidla areálem univerzity, 
plánování cesty, a zajištění bezpečnosti provozu. Pro tento účel řídící kód využívá vstupy z GPS, kamer, Lidaru, 
a ovládací jednotky vozidla. Analýza obrazu a dat naměřených Lidarem není předmětem této práce, výsledek této analýzy 
je získáván z API komponenty určené právě k tomuto účelu, taktéž vyvíjené jako součást projektu.

Software, který je vyvíjen jako součást této práce, je tvořen v Pythonu, a je rozdělen do více komponent. Kromě řídícího
softwaru byl vyvinut také jednoduchý simulátor, umožňující testování řízení. Dále řídící software obsahuje funkční 
vizualizér, pomocí kterého je možné sledovat pohyb vozidla. Při tvorbě řídících komponent byla využita řada open source
knihoven a veřejných zdrojů, včetně map OpenStreetMap nebo Mapy.cz. Zdrojový kód, který je součástí této práci, je 
rovněž otevřený a veřejně dostupný na Githubu.

\section{Architektura řídícího softwaru}
Jak již bylo zmíněno, řídící software je rozdělen do dvou komponent, komunikujících mezi sebou pomocí protokolu TCP na bázi 
architektury klient-server. Jako server slouží program zajišťující přenos dat na rozhraní CAN. Server takto abstrahuje 
nízkoúrovňové a bezpečnostní záležitosti, mezi které patří výpočet kontrolních součtů, zasílání CAN zpráv, TX/RX synchronizace, 
žádání o možnost ovládat vozidlo, regulace rychlosti pomocí PI regulátoru, nebo také zpracování výstupu z GPS modulu. Klient tak
může implementovat pouze samotné plánování cesty, v dotazech na server stačí žádat pouze o dodržování rychlosti, úhlu natočení 
kol, případně aplikace nouzové brzdy. Server samozřejmě dokáže poskytovat zpětnou vazbu, tedy informace o aktuální rychlosti,
pozici, natočení kol, nebo zdraví systému (tzv. \emph{healthcheck}). Server je většinou spouštěn na Raspberry Pi, ačkoliv může
být spouštěn na libovolné jednotce připojené k rozhraní CAN samořiditelné platformy, za předpokladu, že je na dané jednotce
nainstalován interpreter jazyka Python.

Toto dělení nejen zvyšuje úroveň abstrakce, a tím i čitelnost kódu, ale zároveň umožňuje spustit program plánující 
cestu vozidla na vzdáleném stroji. Takto můžeme například program spouštět na výkonnějším stroj a vyhnout se omezením 
Raspberry Pi, virtualizovat ovládací software, nebo implementovat jiný typ klienta, které v současné době existují tři.
Jeden klient se snaží vozidlo řídit na základě vstupů ze senzorů umístěných na vozidle a z komponenty určené pro analýzu obrazu. 
Druhý klient umožňuje řídící pokyny zadávat manuálně do jednoduchého textového prostředí. Třetí klient poté pouze pošle 
předdefinovanou testovací sekvenci. Druhý a třetí klient jsou určeny pouze k ověření funkčnosti komunikace pomocí sběrnice CAN. 
Virtualizace ovládacího softwaru také umožňuje zautomatizovat monitorování ovládání. Ačkoliv k žádným pokusům o automatickou 
orchestraci a monitorování zatím nedošlo, bylo by možné například spouštět řídící software v Kubernetes, nastavit kontinuální 
nasazování pomocí CI/CD pipeline, a logy či metriky kontrolovat strojově pomocí nástrojů jako ELK stack nebo Prometheus.

\section{Server a CAN komunikace}

\subsection{Sběrnice CAN}

Controller Area Network, zkráceně CAN, je standard poskytující určitý základ pro komunikaci na sběrnici. Standard definuje
fyzické požadavky na sběrnici a formát zpráv. Součástí zprávy je osm bytů vyhrazených pro data. Sběrnice CAN je, mimo jiné, hojně
využívána právě v automobilovém průmyslu.

\subsubsection{Vlastní specifikace}
Standard sice přiřazuje každé zprávě pole o velikosti osm bytů vyhrazené pro data, již však nedefinuje, jaká data mají zprávy 
obsahovat, v jakém formátu mají být data posílána, nebo jak se s nimi má pracovat. Obsah zpráv tak musel být vydefinován vlastní 
specifikací, která prošla hned několika iteracemi, a jejíž finální podoba je výsledkem úzké spolupráce konstruktérů fyzické 
platformy a tvůrců řídícího softwaru. Specifikace byla navrhována tak, aby umožnila dostatečně jemné řízení vozidla, poskytovala 
řídícímu softwaru všechna potřebná jízdní data a zahrnovala množství prostředků pro kontrolu integrity zprávy, rychlé zachycení 
chyb či reportování problémů. Na chyby nebo problémy je samozřejmě důležité reagovat, v krajních případech až nouzovým zastavením
vozidla. Software pro ovládání platformy byl již několikrát přepisován na základě upravené specifikace, nejaktuálnější verze 
software však implementuje nejnovější verzi specifikace.

Specifikace, kromě formátu a obsahu dat, určuje také formát, jakým je kódována rychlost vozidla nebo natočení předních kol. Tyto
hodnoty jsou kódovány a uloženy v jednom bytu. Při práci s CAN sběrnicí tak je nutné data správně přepočíst a zakódovat, případně
dekódovat a převést do formátu IEEE 754.

\subsection{TCP server a jeho metody}

Za účelem zvýšení abstrakce a zjednoduššení ovládání vozidla, ale také aby bylo umožněno vzdálené ovládání platformy
a zjednodušila se výměna způsobů ovládání, je nízkoúrovňové řízení CAN zprávami abstrahováno a obaleno jednoduchým TCP serverem.
Při vývoji serveru se přirozeně nabídla možnost implementace REST API a využití protokolu HTTP, nebo využití jednoho z rozšířených
RPC prokolů, jako jsou XMLRPC nebo gRPC. Ačkoliv validní, tyto způsob, implementace se jevily jako neefektivní -- rozhraní 
poskytované serverem je velmi jednoduché, plaintextový protokol HTTP nebo variace protokolů RPC tak působily až zbytečně 
komplikovaně. Pro komunikaci byl nakonec použit vlastní binární protokol a jeho implementace přímo nad TCP sockety. 
Výhodou je rychlost a jednoduchost implementace, úspornost a výkonnost řešení v reálném provozu, nebo fakt, že toto řešení 
nevyžaduje žádné externí závislosti nad rámec standardní knihovny jazyka Python. Nevýhody současného řešení by se projevily 
pouze při výrazném nárůstu komplexity poskytovaného rozhraní. Tento scénář by mohl vést k znepřehlednění současného řešení. 
Nutnosti zvyšování komplexity rozhraní ovšem v současné době nic nenasvědčuje. 

Aktuální verze serveru implementuje celkem pět metod. Na vstupu přijímá celkem tři byty. První byte je vyhrazen pro identifikaci 
metody serveru. Následující dva byty dotazu obsahují data. Klient musí vždy poslat minimálně tři byty, na vstupu jsou, nezávisle
na metodě, pokaždé očekávány tři byty. Jestliže klient pošle v dotazu více než dva datové byty, budou nadbytečná data ignorována.

Jako identifikátor metody je využito celé nezáporné číslo. V případě, že první byte zprávy obsahuje identifikátor nepříslušící 
žádné metodě, je dotaz ignorován. Obsah datových bytů je určen danou metodou serveru. Pokud pro danou funkcionalitu není třeba 
předávat serveru žádná data, může být obsah datových bytů libovolný, data budou ignorována. 

\subsubsection{Metoda \emph{drive}}
Metoda \emph{drive} slouží k předání požadované rychlosti vozidla a natočení kol. Obsah první datového bytu je interpretován jako 
rychlost, druhý datový byte reprezentuje natočení předních kol. Oba datové byty jsou interpretovány jako celá znaménková čísla. 

Fyzická změna rychlosti nebo natočení kol samozřejmě nemůže nastat okamžitě, rychlost změny je omezena fyzikálními vlastnostmi 
konstrukce i PI regulátorem, proto nelze vrátit klientu potvrzení o provedené změně. Odpovědí metody je ekvivalentní s metodou 
\emph{healthcheck}. \emph{Healthcheck} v odpovědi slouží jako potvrzení přijetí dotazu a schopnosti dotazu vyhovět. V případě, 
že klient potřebuje potvrzení o dosažení požadované rychlosti nebo úhlu natočení kol, je možné po zavolání metody \emph{drive} 
opakovaně volat metodu \emph{info}.

\subsubsection{Metoda \emph{healthcheck}}
Metoda \emph{healthcheck} slouží k ověření \emph{živosti} systému. \emph{Živost} je binární stav, systém je buď \emph{živý}, nebo
\emph{mrtvý}. Server je považován za \emph{živý}, jestliže je schopen zpracovávat TCP spojení, a zároveň má plnou kontrolu 
nad vozidlem. V opačném případě je server považován za \emph{mrtvý}. Metoda nepřijímá žádný vstup, datové byty jsou ignorovány.

\emph{Healthcheck} ve své odpovědi posílá pouze jeden byte. Pokud je server živý, obsahuje odpověď nenulovou hodnotu. V opačném
případě vrátí server nulu, nebo neodpoví vůbec.

\subsubsection{Metoda \emph{info}}
\emph{Info} metoda TCP serveru se využívá k získání aktuálních jízdních dat vozidla. Na vstupu nepřijímá žádné parametry, všechny
datové byty jsou tedy ignorovány. Na výstupu server vrátí tři byty. První dva byty reprezentují znaménková celá čísla. Obsahem
prvního bytu je aktuální rychlost vozidla. Druhý byte obsahuje úhel natočení kol. Poslední byte poté obsahuje logickou hodnotu
značící stav nouzové brzdy. Jestliže je hodnota třetího bytu nulová, nouzová brzda není aktivována. Nenulová hodnota značí, 
že došlo k aktivaci nouzového brždění, a bude třeba jej deaktivovat, pokud chceme pokračovat v jízdě.

\subsubsection{Metoda \emph{ebrake}}
\emph{Ebrake}, nebo-li \emph{'Emergency Brake'}, česky 'nouzová brzda', slouží k ovládání nouzového brždění. První byte vstupu
je serverem interpretován jako logická hodnota značící požadovaný stav nouzové brzdy. Nenulovou hodnotu server považuje za signál
k aktivaci nouzového brždění. Při přijetí nulové hodnoty na vstupu poté dochází k deaktivaci nouzové brzdy. Druhý byte vstupu 
je ignorován.

Výstup metody \emph{ebrake} je opět ekvivalentní výstupu metody \emph{healthcheck}. Nenulová hodnota na výstupu značí, že server
požadavek přijal a je schopen jej zpracovat. Nula reprezentuje stav, kdy server nemá kontrolu nad vozidlem a není schopen 
nouzovou brzdu aktivovat. Pokud však ke ztrátě kontroly došlo důsledkem chyby, vozidlo začne nouzově brzdit samo.

\subsubsection{Metoda \emph{position}}

\emph{Position} využijeme, chceme-li zjistit geografickou polohu vozidla. Tato metoda, která neakceptuje žádný vstup, ve svém 
17 bytovém výstupu vrací tři hodnoty. 

První byte obsahuje logickou hodnotu značící, zda je geografická poloha věrohodná. Nulový byte značí polohu nevěrohodnou, 
nenulový byte věrohodnou. Poloha je považována za nevěrohodnou např. pokud nedojde ke správné inicializaci vozidla, GPS modul 
nedokáže zaměřit pozici, apod. 

V následujících osmi bytech je zakódována zeměpisná šířka (latitude), v posledních osmi bytech pak zeměpisná délka (longitude). 
Ačkoliv se jedná o desetinná čísla, pro zjednodušení jsou zakódována jako čísla celá. Skutečná hodnota zeměpisné šířky a délky 
je vynásobena číslem $10^{10}$ a převedena na celé číslo. Výsledné číslo je poté odesláno ve formátu little endian. Při zpracování
odpovědi serveru si tedy klient musí dát pozor na endianitu a správně dekódovat přijatou hodnotu.

\subsection{Regulace PI regulátorem}
Elektrický motor vozidla samozřejmě koncept rychlosti nezná. Elektrický motor je schopen vydávat určitý točivý moment, o který
je samozřejmě možné si zažádat. Mezi točivým momentem a rychlostí vozidla ovšem neexistuje lineární závislost. Rychlost vozidla
může ovlivnit více faktorů. Např. při vysokém konstatním točivém momentu na vodorovné nebo svažující se vozovce by vozidlo mohlo 
neustále zrychlovat, minimálně dokud by nedosáhlo rychlosti nebezpečné nejen pro samotnou platformu, ale především pro své okolí. 
Naopak při stoupání by vozidlo mohlo na rychlosti ztrácet.

O vydání určitého točivého momentu je možné požádat za pomoci CAN sběrnice. Řídící software vyšle CAN zprávu s požadovanou
rychlostí. Kontrolní jednotka vozidla zprávu zpracuje, zkontroluje, a předá požadavek elektromotoru, který upraví svůj výstup.
Při ovládání vozidla ovšem není určování momentu vhodné. Podstatně jednodušší by bylo žádat o dodržování konkrétní rychlosti, 
ne momentu. Proto je točivý moment třeba regulovat. A právě zde se jako řešení nabízí využití PI regulátoru.

Regulátor je zabudován do zdrojového kódu serveru a je implementován softwarově. TCP server na vstupu metody \emph{drive} dostane
požadavek na udržování rychlosti. Server metodu zpracuje a předá regulátoru jako požadovanou hodnotu. Druhým vstupem regulátoru
je aktuální rychlost, kterou lze nalézt v CAN zprávách kontrolní jednotky platformy. Na svém výstupu regulátor vrací požadovaný
moment, který řídící program předá vozidlu pomocí sběrnice CAN.

Proporcionální složka regulátoru slouží ke změně točivého momentu na základě odchylky naměřené rychlosti od rychlosti požadované.
Integrační složka regulátoru slouží ke sčítání předchozích odchylek a k udržování určité hladiny momentu, aby např. při dosažení
požadované rychlosti, kdy odchylka bude rovna nule, nedošlo k vypnutí motoru. Derivační složka není v této aplikaci potřeba, proto
je využit pouze PI regulátor.

Regulováno je i natočení kol. Zde je regulátor využit zejména k zabránění velkých skokových změn v požadavku na natočení kol. Není
žádoucí, aby požadovaný úhel například rychle osciloval mezi hodnotami -20 a 20 stupňů. PI regulátor zde slouží právě k zajištění 
pozvolné změny požadavku. Požadovaný úhel natočení kol je na CAN sběrnici zadáván ve stupních, aplikace regulátoru je zde velmi 
jednoduchá.

\subsection{Python implementace}

Pro implementaci kódu byl zvolen jazyk Python 3.6. Server má pouze dvě externí závislosti, a to knihovny \emph{python-can} a
\emph{gpsd-py3}. Dále je samozřejmě extenzivně využívána standardní knihovna jazyka Python.

\subsubsection{Konfigurace}
Ke konfiguraci komponenty jsou využívány systémové enviromentální proměnné, a to zejména kvůli jednoduchosti řešení a konzistence 
s ostatními komponentami. Jelikož je program spouštěn pouze na Raspberry Pi fyzicky připojenému ke CAN sběrnici vozidla,
k virtualizaci pomocí orchestračních nástrojů, jako je třeba Kubernetes, pravděpodobně nikdy nedojde. Argument o jednoduchosti
konfigurace za využití proměnných prostředí při virtualizaci tak u této komponenty neplatí.

\subsubsection{TCP server}
K vytvoření TCP serveru byl využit modul \emph{socketserver}, který náleží ke standardním modulům patřícím do jazyka Python.
Server funguje na synchronní bázi a na zpracování probíhá pouze na jednom vlákně. K serveru je v jednom momentě připojen vždy jen
jeden klient, jenž posílá pouze relativně malé množství dotazů. Asynchronní nebo vícevláknová implementace tak není potřeba, 
pouze by zapříčinila znepřehlednění kódu, a to s velmi nízkými až dokonce žádnými benefity. Server naslouchá na konfigurovatelném
portu. Instanci rozhraní vozidla nebo port lze serveru předat parametrem při inicializaci. Pokud tak uživatel neučinní, rozhraní
vozidla je vytvořeno automaticky a číslo portu je vyčteno z enviromentální proměnné \emph{SERVER\_PORT}.

\subsubsection{Rozhraní vozidla}
Rozhraní vozidla, v kódu třída \emph{Car}, představuje abstrakci nad sběrnicí CAN, a to aby samotný TCP server nemusel pracovat
přímo s CAN zprávami. Instance třídy \emph{Car} umožňují TCP serveru nastavovat nebo získávat hodnotu úhlu zatočení předních kol,
rychlost, záchrannou brzdu, případně získat GPS pozici vozidla, aniž by TCP server musel počítat s momentem, CAN hodnotami, nebo
pracovat s hardwarovými moduly.

Za účelem práce s CAN sběrnicí byly implementovány třídy \emph{Transmitter} a \emph{Receiver}, které slouží k přijímání a vysílání
zpráv. Interně třídy využívají prostředky poskytované modulem \emph{python-can}, zvenčí představují především abstrakci 
nad nízkoúrovňovými CAN záležitostmi. \emph{Transmitter} na sběrnici vysílá zprávy \emph{DriveMessage} a \emph{ControlMessage}.
\emph{Receiver} přijímá zprávu \emph{CarData}. Data po přijetí zpracuje a předá za pomocí callbacku třídě \emph{Car}. Ke konverzi
mezi CAN hodnotami a formátem IEEE 754 slouží pomocné třídy \emph{Driving} a \emph{Steering}. GPS souřadnice jsou z fyzického GPS
modulu připojeneho k Raspberry Pi získávány za využití modulu \emph{gpsd-py3}.

CAN zprávy jsou posílány periodicky. K zajištění periodicity je využita knihovna \emph{python-can}, která zvládne periodické
zasílání zpráv zajistit. Zasílaná data lze libovolně měnit za využití rozhraní abstraktní třídy \emph{ModifiableCyclicTaskABC} 
poskytované modulem \emph{python-can}. Knihovna dokáže zajistit pravidelné opakované vysílání zpráv na separátním vlákně. V kódu
tak stačí vytvořit periodickou úlohu a pouze podle potřeby upravovat požadovaná jízdní data či kontrolní součty.

\section{Simulátor vozidla}

Simulátor vozidla, taktéž implementovaný v Pythonu, má za cíl nahradit a nasimulovat server běžící na Raspberry Pi připojeném
k autonomní platformě. Simulátor implementuje naprosto stejné API jako skutečný server. Fyzické vozidlo je však nahrazeno pouhou
virtuální aproximací. Simulace samozřejmě určitým způsobem počítá s fyzikálními reáliemi vozidla -- natočení přední soupravy 
určitou dobu trvá, akcelerace není lineární, vozidlo samovolně zpomaluje, pokud motor neposkytuje dostatečný výkon. Přesto je 
simulace zjednodušená a počítá pouze s elementárními fyzikálními zákony. Simulátor vznikl s cílem umožnit jednoduché a rychlé 
testování jednotlivých částí řídícho softwaru. Realistická a přesná simulace by byla pro tento účel příliš složitá a silně 
překračovala rozsah a cíl projektu.

Program po spuštění vytvoří TCP server a inicializuje vozidlo ve výchozí pozici. Na základě vstupu TCP serveru je poté vozidlo 
ovládáno. Geografická poloha simulované platformy je automaticky aktualizována s pohybem, aby výstup metody \emph{drive} TCP 
serveru odpovídal reálnému výstupu v praktickém zapojení. Klientu je tak poskytnut ekvivalent skutečného vozidla. 

Kód simulátoru nemá žádné závislosti. Využívá pouze standardní knihovnu jazyka Python verze 3.6. Ke konfiguraci lze využít 
proměnné prostředí systému. V konfiguraci lze specifikovat port, na kterém má TCP server naslouchat, zapnout debugovací funkce, 
nebo také nastavit výchozí pozici vozidla.

Simulátor je využíván například při testování práce s mapovými podklady, jako je vyhledávání a plánování cesty, hledání nejbližší
využitelné vozovky, při vývoji a testování vizualizéru pohybu vozidla, nebo při vývoji uživatelského rozhraní určenému k ovládání
vozidla, tedy když realistická simulace není nutná a jednoduchá aproximace více než stačí.

\section{Klient a ovládací software}

\subsection{Řízení vozidla}

\subsection{Komunikace se serverem}

\subsection{Komunikace s kamerami}

\subsection{Vizualizér}

\subsection{Ovládání vizualizérem}

\subsection{Python implementace}

\section{Testované způsoby řízení vozidla}

\subsection{Řízení vozidla pomocí GPS}

\subsection{Řízení vozidla pomocí detekce grafických kódu}

\subsection{Řízení vozidla kombinací vstupů z GPS, kamery, a za využití mapových podkladů}

\section{Práce s mapovými podklady}

\subsection{Načítání mapových podkladů}

\subsection{Vyhledání aktuální pozice na mapě}

\subsection{Plánování cesty za využití mapových podkladů}

\section{Budoucí vývoj vozidla}

\subsection{Vstup z Lidaru}

\subsection{Chytřejší práce s mapovými podklady}

\subsubsection{Automatická aktualizace mapových podkladů}

\subsubsection{Cena cesty odvíjející se od denní doby}

\subsection{Ovládání vozidla pomocí webové aplikace}

\section{Závěr}

\section{Technické detaily}

\subsection{Křížové odkazy}
\label{sec:CrossReferences}
Odborné texty, mezi které lze počítat i bakalářské, diplomové a disertační práce, obvykle obsahují množství křížových odkazů odkazující na nejrůznější části textu:
\begin{description}
	\item [kapitoly] -- například odkaz na kapitolu \ref{sec:Uherske}. Pokud odkazujeme na kapitolu, která je značně vzdálená od současné stránky, bývá dobrým zvykem k odkazu na číslo kapitoly přidat ještě i odpovídající číslo stránky, jako například pokud odkazujeme na kapitolu \ref{sec:Introduction} na straně \pageref{sec:Introduction}.

	\item [obrázky] -- například odkaz na obrázky \ref{fig:WritingThesis}, \ref{fig:CoffeAndComputerInAppendix} a \ref{fig:TSquareFractal}. Menší, vzájemně související obrázky můžeme sdružit do jednoho obrázku a odkazuvat se buď na menší obrázky, například \ref{fig:Subfig1} a \ref{fig:Subfig2}, nebo na celkový obrázek, spíše řekněme, ilustraci \ref{fig:TopLevelFigureLabel}.

	\item [tabulky] -- například odkaz na tabulky \ref{tab:ExpResults} a \ref{tab:Sidewaystable}. Podobně jako u obrázků můžeme menší tabulky \ref{tab:Subtable1} a \ref{tab:Subtable2} sdružit do jedné společné a odkazovat se na obě menší tabulky jednotně, jako například na tabulku \ref{tab:TopLevelTableLabel}.

	\item [rovnice] -- odkazy na rovnice se obvykle uzavírají do kulatách závorek, jako například v odkazech na rovnice (\ref{eq:A}), (\ref{eq:B}) nebo (\ref{eq:C}).

	\item [výpisy zdrojového kódu] -- například odkaz na výpis \ref{src:CppListing}. Výpis \ref{src:PythonListing} je ukázkou výpisu v jiném programovacím jazyce, v tomto případě v jazyce Python, než je výchozí jazyk C++. Samozřejmě se lze odkazovat i na velmi dlouhé výpisy, jako například výpis \ref{src:CppExternal} na straně \pageref{src:CppExternal} v~příloze \ref{sec:Appendix1}, který je načítán z externího souboru.
\end{description}

\subsection{Jak citovat}
Obecně lze říci, že pro bibliografické odkazy a citace dokumentů používáme zásadně normu ČSN ISO 690.
\subsubsection{Odkaz v textu}
Pro odkazy v textu používáme číselné označení citací dokumentů ohraničené hranatými závorkami. Takže například můžeme citovat časopisecké \emph{články} \cite{herrmann, bertram, moore, yoon, sigfridsson, baez/article}, \emph{knihy} \cite{wilde, nietzsche:ksa1, averroes/bland, hammond, cotton, knuth:ct:a, gerhardt, gonzalez, companion}, \emph{periodika} \cite{jcg}, \emph{bakalářské, diplomové či diserteční práce} \cite{geer}, \emph{patenty} \cite{kowalik, almendro, sorace, laufenberg}, \emph{online zdroje} \cite{ctan, wassenberg, itzhaki, markey, baez/online} či \emph{manuály} \cite{cms}.

\subsubsection{Seznam citací}
Seznam citací je umístěn na konci závěrečné práce, před přílohami, a musí obsahovat všechny citace na které je v textu práce odkazováno.

\subsection{Překlad}
Kompilaci této ukázkové práce je možné provést pomocí několika volání pdf\LaTeX{}u a programu Biber v následujícím pořadí:
\begin{verbatim}
pdflatex main
biber main
pdflatex main
pdflatex main
\end{verbatim}

\printbibliography[title={Literatura}, heading=bibintoc]

\appendix
\section{Plné tkví drah pokles průběhu}


\end{document}
